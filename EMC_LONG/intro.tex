\section{Introduction} \label{intro.sec}

Quantum chromodynamics (QCD) is the theory governing the strong interaction,
with quarks and gluons as elementary degrees of freedom. The interaction
between quarks is mediated by gluons as the gauge bosons. Understanding QCD in
terms of the elementary quark and gluon degrees of freedom remains the
greatest unsolved problem of strong interaction physics. The challenge
arises from the fact that quarks and gluons cannot be examined in isolation.
The degrees of freedom observed in nature (hadrons and nuclei) are different
from the ones typically used in the QCD formalism (quarks and gluons).
However, detailed studies of the structure of hadrons, mainly protons and
neutrons, provide a wealth of information on the nature of QCD. Thus, one of
the main goals of the strong interaction physics is to understand how the
fundamental quark and gluon degree of freedom give rise to the nucleons and to
inter-nucleon forces that bind nuclei.
 

The investigation of deep-inelastic scattering of leptons off the nucleon is
one of the most effective ways for obtaining fundamental information on the
quark-gluon substructure of the nucleon. Nuclear structure functions probe the
impact of the nucleon binding and motion in the nucleus, as well as possible
modification to the structure of a nucleon in the nuclear medium. Measurements
by the European Muon Collaboration~\cite{aubert83} showed the unexpected
result that the nuclear structure function differed significantly from the
sum of proton and neutron distributions.  This observation was dubbed the
`EMC effect', and is still the focus of experimental and theoretical efforts
to understand the origin of these differences in detail.  We decribe here
an experiment where electrons were scattered off from proton and several other
nuclear targets to better understand the possible modification of hadron
properties in the nuclear environment, with a focus on light nuclear targets.

This article is structured as follows. In the next section we briefly
discuss electron scattering, structure functions and introduce the
kinematics. In section~\ref{intro.sec} we discuss the EMC effect and do a short
survey of the findings of earlier experimental and theoretical investigations
and briefly discuss the physics motivation behind the present
experiment. Section~\ref{apparatus.sec} gives an overview of the experimental
apparatus used to collect the presented data. Section~\ref{anal.sec} describes
the data analysis procedures and section~\ref{syst.sec} discusses the details
of the systematic uncertainties. The final results are presented in
section~\ref{results.sec} with conclusions and an overview of the results
given in section~\ref{concl.sec}.



\subsection{Kinematics and definitions} \label{kinem.ssec}

Consider electron scattering off a stationary target nucleon through the
exchange of a single virtual photon,
%
\begin{equation}
e^{-}(k) + N(P) \longrightarrow e^{-}(k^{'}) + X ~,
\end{equation}
%
where $k$ and $k^{'}$ are the four momenta of the initial and scattered
electrons and $P$ is the four momentum of the target nucleon. The four
momentum of the incoming electron is $k= ( E, \overrightarrow k ) $ and of the
scattered electron is $k= ( E^{'}, \overrightarrow {k ^{'}}) $. Since the target
is at rest in the laboratory frame its four momentum is $P = ( M,
\overrightarrow 0 ) $ where $M$ is the nucleon rest mass. Experimentally, the
produced hadrons $X$ are not detected in inclusive scattering of electrons
with a fixed energy, $E$. Only the scattered electron energy $E^{'}$ and the
scattering angle $\theta$ relative to the incident beam. The scattering
process takes place through the electromagnetic interaction by the exchange of
a virtual photon $\gamma^{*}$, with energy, $\nu=E - E^{'}$ and momentum
$\overrightarrow q$. In the laboratory frame and ignoring the electron mass,
one can express the negative of four momentum transfer square of the
virtual photon exchanged in the scattering process as
$Q^2=4EE^{'}\sin^2\left( \theta/2\right) $ and the invariant mass of the
final hadronic system as $W=\sqrt{M^2+2M\nu - Q^2}$. The Bjorken scaling
variable, $x=Q^2/2M\nu$, represents the longitudinal momentum fraction of the
hadron carried by the interacting parton in the infinite momentum frame. For
electron scattering from a free nucleon, $x$ ranges from 0 to 1. For scattering
from a nucleus of mass number $A$, the $x$ range goes from 0 to
$M_A/M \approx A$.

In terms of the deep-inelastic structure functions $F_1(x,Q^2)$
and $F_2(x,Q^2)$, the differential cross section for scattering of an
unpolarized electron in the laboratory frame can be written as
%
\begin{eqnarray} \nonumber \label{xsec_eqn}
\frac{d^2\sigma}{d\Omega dE^{'}} &=& \frac{4\alpha^2E^{'2}}{Q^4} \cos^2(\theta/2) \big[ F_2(x,Q^2)/\nu \\
&&+ 2\tan^2(\theta/2) F_1(x,Q^2)/M \big] ~,
\end{eqnarray}
%
where $\alpha$ is the fine structure constant. For
brevity, this doubly differential cross section is denoted by the symbol
$\sigma$. When $Q^2$ and $\nu \rightarrow \infty$, the structure functions
will only depend on the ratio $Q^2/\nu$ or equivalently on the variable $x$
\cite{bjorken_scaling}. Thus, in this scaling region the structure functions
are simply a function $x$. In the quark parton model (QPM), this scaling
behavior is due to the elastic scattering of the quarks inside the nucleon. In
this model, the structure function $F_2$ is given by
%
\begin{eqnarray} \label{qpmf2_eqn}
 F_2(x)= \sum_f e_f^2 x q_f(x) ~.
\end{eqnarray}
%
where the distribution function $q_f(x)$ is the expectation value of the
number of partons of flavor $f$ (up, down, strange...) in the hadron, whose
longitudinal momentum fraction lies within the interval $[x, x+dx]$ and $e_f$
is the charge of the parton, in units of electron charge.

Experimentally this scaling is not exact.  In the region of deep-inelastic
scattering (DIS), the structure functions do not scale exactly, and instead
depend logarithmically on $Q^2$.  This is a consequence of QCD, in which the
parton distribution functions (pdfs) are not scale independent, but evolve
with $Q^2$.  So the logarithmic scaling violations associated with QCD
do not break down the connection between the structure function and the
underlying pdfs, but simply reflect the scale-dependence of the pdfs.

Along with the $Q^2$ dependence associated with QCD, additional power
corrections appear at lower $Q^2$ values, mainly at large $x$.  So-called
"target mass corrections"~\cite{Schienbein_tarmass_rev} yield deviations from
scaling at finite $Q^2$ values arising from terms neglected in the high-$Q^2$
approximations used in the ideal scaling limit.  In addition, higher-twist
effects, associated with breakdown of the assumption of incoherent elastic
scattering from individual quarks at lower $Q^2$, also modify the scaling
behavior.  This is most clearly manifested in the appearance of clear
structures in the inclusive structure function associated with production of
individual resonances.

Analogous to the absorption cross section for real photons, the $ F_1$ and $
F_2$ structure functions can be expressed in terms of longitudinal
($\sigma_L$) and transverse ($\sigma_T$) virtual-photon cross sections
%
\begin{equation}\label{rosenrxsec_eqn}
 \frac{d^2\sigma}{d\Omega dE^{'}} =
\Gamma \left[ \sigma_T \left( x, Q^2\right) +
\epsilon \, \sigma_L \left( x, Q^2\right) \right]~,
\end{equation}
%
where $\epsilon =\Gamma_L/\Gamma_T= \big[1+2~(1+Q^2/4M^2x^2)
\tan^2\frac{\theta}2\big] ^{-1}$ is the virtual polarization parameter,
$\Gamma$ is the virtual photon flux, and $\Gamma_L$ and $\Gamma_T$ defines the
probability that a lepton emits a longitudinally or transversely polarized
virtual photon.


The ratio of longitudinal to transverse virtual-photon absorption cross
section is given by
%
\begin{eqnarray} \label{r_eqn}
R(x,Q^2) = \frac{\sigma_L}{\sigma_T} = \left[ \left(1+\frac{\nu^2}{Q^2} \right)
\frac{M}{\nu} \frac{F_2(x,Q^2) }{F_1(x,Q^2) }\right]-1 ~.
\end{eqnarray}
%
Using equations~\ref{xsec_eqn} and~\ref{r_eqn}, the per-nucleon cross section
(cross section divided by the total nucleon number) ratios for two different
nuclei $A_1$ and $A_2$ can be written as
%
\begin{eqnarray} 
\frac{\sigma_{A_1}}{\sigma_{A_2}} =
\frac{F_2^{A_1}\, (1+\epsilon \, R_ {A_1}) \, (1+ R_ {A_2}) }
     {F_2^{A_2}\, (1+\epsilon \, R_ {A_2}) \, (1+ R_ {A_1}) }~. 
\end{eqnarray}
%
Note that when $\epsilon =1$ or $R_{A_1} =R_{A_2}$, the ratio of the $F_2$
structure functions is identical to the cross section ratio.

%In that case
%%
%\begin{eqnarray} 
%\frac{\sigma_{A_1}}{ \sigma_{A_2}} = \frac{F_2^{A_1}}{F_2^{A_2}} ~. 
%\end{eqnarray}
%%
%In other words if $\epsilon=1$ or $R_{A_1} =R_{A_2}$, then the nuclear
%dependence of the structure function at a given kinematics is directly related
%to the ratio of cross sections measured at the same kinematics.

In this and all previous extractions of the EMC effect, it is assumed that $R$
is target independent, and therefore the cross section ratios correspond to the
$F_2$ structure function ratios.  Because the structure functions depend
on $Q^2$, this ratio may also have a $Q^2$ dependence which we must account
for in comparing our data to measurements at other different $Q^2$ values.
However, the effect of QCD evolution on the ratios should be essentially
negligible as the evolution is nearly identical for all nuclei, and so cancels
in the ratio.  The only significant contribution to the ratio from target mass
corrections corresponds to a simple change of variables from Bjorken-$x$ to
Nachtmann-$\xi$ when comparing measurements at different $Q^2$
(Sec.~\ref{hixcor.ssec}).  Thus, in kinematics where any remaining higher-twist
contributions are small or $A$ independent, the comparison of EMC ratios from
experiments at different $Q^2$ values is straightforward. Accounting for this
change of variables mentioned above, it has been shown that the EMC ratios are
independent of $Q^2$ down to very low values of $Q^2$ and $W^2$, well below
the typically-defined DIS regime~\cite{Arrington:2003nt, niculescu06}.


\subsection{EMC effect}\label{emc.ssec}

Nuclei consist of protons and neutrons bound together by the strong nuclear
force, with binding energies of 1--2\% of the nucleon mass, and characteristic
momenta below 200--300 MeV/c. Because DIS involves incoherent scattering from
the quarks, and the energy and momentum scales associated with nuclear binding
are small compared to the external scales in DIS, the naive assumption was
that the nuclear structure function in high-energy scattering from a nucleus
with $Z$ protons and $N$ neutrons would simply be the sum of the proton and
neutron structure functions:
%
\begin{equation} \label{f2asimple1_eqn}
F^A_2(x,Q^2) = Z F^{p}_2(x,Q^2) + N F^{n}_2(x,Q^2) ~.
\end{equation}

%%
%or in terms of the parton distribution functions in the QPM,
%%
%\begin{equation} \label{f2asimple2_eqn}
%q^A_f (x,Q^2) =\frac{Z}{A}q^{p}_f(x,Q^2) + \frac{(A-Z)}Aq^{n}_f(x,Q^2) ~.
%\end{equation}
%%
%Here, $F^{p}_2(x,Q^2)$ and $F^{n}_2(x,Q^2)$ are nucleon structure functions
%and $q^{n}_f(x,Q^2)$ and $q^{n}_f(x,Q^2)$ represent the quark, anti-quark
%and gluon densities in the nucleons.

While the typical scale of the Fermi momentum is small compared
to the momentum scale of the probe, the longitudinal component is directly
added to the momentum of the virtual photon and cannot be completely 
neglected. It is necessary to perform a convolution of the pdfs of the proton
and neutron with the momentum distribution of the nucleons in the
nucleus~\cite{Akulinichev:1985xq}:
%
\begin{equation}\label{conv_eqn}
F_2^A(x)=\int_{x}^A dz\, f_{N}^A(z,\epsilon) F_2^{N}\left(\frac{x}{z}\right) ~,
\end{equation}
%
where the longitudinal momentum distribution function $f_{N}^A(z,\epsilon)$
for the nucleon is given by,
%
\begin{equation}\label{conv2_eqn}
f_{N}^A(z,\epsilon)=
\int d^{4}p~S_N(p)~\delta\left(z-\left(\frac{pq}{M_N q_{0}}\right) \right) ~.
\end{equation}
%
Here, $S_N(p)$ is the spectral function of the nucleus (here assumed to be
identical for protons and neutrons), $z$ is the light-cone
momentum carried by the nucleon and $\epsilon$ its removal energy. The
four-momenta of the struck nucleon and virtual photon are given by $p$
and $q$, where $q_{0}$ is the energy transferred by the virtual photon. One
can think of the convolution as ``smearing'' the nucleon pdf in $x$, yielding
little change where the pdf is relatively flat in $x$, and larger effects
where it grows or falls rapidly. Calculations showed that the effect was
minimal at low $x$ values, but that the convolution has a large impact for $x
\gtorder 0.6$, where the pdf of the nucleon falls rapidly~\cite{bodek81a,
bodek81b, frankfurt81}.


Therefore, it came as a surprise when this expectation was shown to be
incorrect by measurements which showed significant effects on the nuclear
pdf for nearly all values of $x$~\cite{aubert83}. As part of a
comprehensive study of muon scattering, the European Muon Collaboration
compared data from iron with data from deuterium by forming a per-nucleon
structure function ratio of these targets. Since the $x$ distributions of up
and down quarks differ, yielding different structure functions for the
proton and neutron, EMC ratios are usually taken as a ratio of a heavy
isoscalar target to deuterium. This cancels out the contribution due to
the difference between the proton and neutron structure function but
yields a ratio which depends on the nuclear effects in both the heavy
nucleus and the deuteron. For non-isoscalar nuclei, a correction is typically
applied to estimate the effect of the neutron excess in heavy nuclei. Note
that many calculations provide the ratio of the heavy nucleus to the sum of
free proton and neutron structure functions. This provides a more direct
measure of the nuclear effects in the nucleus, but cannot directly be compared
to the data, as the lack of a free neutron target makes direct measurements of
the free neutron structure function impossible.


When plotted as a function of $x$, the EMC ratio shows significant deviation
from unity. The deviation of this ratio from unity was unexpected, and, this
target-mass number dependence in deep-inelastic scattering is known as the
EMC effect. This discovery had a significant impact on our views of the
structure of nuclei, and has spurred discussion of the importance of the
concepts of quarks, gluons and QCD to nuclear physics.


Though the boundaries are somewhat arbitrary, generally the $x$ dependence of
the cross section ratios are divided into four regions in $x$. The gross
features of the data are; the region $x<0.1$, where the nuclear cross sections
are suppressed (known as the shadowing region); the region $0.1<x<0.3$, where
the nuclear cross sections are slightly enhanced compared to nucleon cross
sections (anti-shadowing region); the region $0.3<x<0.7$, where a large
suppression of the nuclear cross section is observed (``EMC effect'' region),
and the region $x>0.7$ where the EMC ratio increases and grows beyond unity
due to the convolution ("Fermi smearing") effects.


\subsection{Previous measurements of the EMC effect}\label{prevexpt.ssec}

After the initial observation of an unexpected nuclear dependence in the
structure functions of heavy nuclei~\cite{aubert83}, measurements were
performed at both CERN~\cite{cern_na2prime1, cern_na2prime2, cern_bcdms1} and
SLAC \cite{slace49b,slace61,slace87,slace139}. Further measurements by the EMC
collaboration~\cite{aubert86} and New Muon Collaboration
()~\cite{Arneodo:1995cs,Amaudruz:1995tq} significantly improved the
precision and kinematic range of measurements at low $x$, mapping out in
detail the shadowing region for a range of nuclei. The HERMES collaboration
also measured DIS cross sections on several nuclear targets including
$^3$He~\cite{hermes_ackerstaff,hermes_correction_airapetian}. The data
in the anti-shadowing region are consistent with unity, while data at
higher $x$ have large uncertainties.

Focusing on the high-$x$ region, SLAC experiment E139~\cite{slace139}
mapped out the EMC effect for $^4$He, Be, C, Al, Ca, Fe,
Ag, and Au in the range $0.09<x<0.9$ and $2<Q^2<15$~GeV$^2$. Examining
the target ratios, and in particular their deviations from unity,
the experiment showed no significant $Q^2$ dependence and an identical $x$
dependence for all nuclei, although the high-$x$ behavior of $^4$He appeared
to differ, but not in a significant fashion given its large uncertainties.
The $A$-dependence of the nuclear effects could be parameterized several
different ways: varying logrithmically with A, linearly with A$^{-1/3}$,
or being proportional to the average nuclear density (assuming a uniform
sphere based on the measured nuclear charge radius). It common to assume that
the EMC effect scales as A$^{-1/3}$~\cite{arrington12c}, based on the local
density approximation~\cite{antonov86}, and this has been used to provide an
extrapolation of the EMC effect to infinite nuclear matter~\cite{sick92}.


The universal $x$ dependence and weak A dependence for heavy nuclei makes
it difficult to evaluate models of the EMC effect~\cite{geesaman95,
Norton_emcreview, sargsian03}.  In addition, the EMC effect at very large $x$
values had not been well measured. The typical DIS requirement, $W^2 >
4$~GeV$^2$, yields extremely high $Q^2$ measurements for $x \gtorder 0.8$,
where the cross sections are extremely small. However, recent extraction of
EMC ratios from JLab experiment E89-008 in the resonance region
($1.2<W^2<3.0$~GeV$^2$ with $Q^2 = 3$--4 GeV$^2$) demonstrated that the
nuclear effects in the resonance region and DIS region are
identical~\cite{Arrington:2003nt}. This implies that relaxing the constraint
on $W^2$ may allow for measurements at larger $x$ values than previously
accessed.  Precise measurements at large $x$ allow for tests of the
convolution model where other effects are expected to be small, providing a
constraint on the convolution effects which must be accounted for at all $x$
values.


\subsection{Theoretical models}\label{theory.ssec}

Even though it has been three decades since the discovery of the EMC effect,
and there are extensive data on its $x$ and $A$ dependence for $A \ge 12$,
there is no clear consensus as to its origin. The EMC effect has been under
intense theoretical and experimental study since the original observation (see
the reviews~\cite{geesaman95, Norton_emcreview, sargsian03} and references
therein). The formalisms used to explain the observed effect ranges from
traditional nuclear descriptions in terms of pion exchange or binding energy
shifts, to QCD inspired models such as dynamical rescaling, multi-quark
clustering and de-confinement in nuclei, some of which involve changes to
the nucleon's internal structure when in the dense nuclear medium.


%convolution models

Traditional calculations begin with the convolution model, where the nucleon
motion and binding modify the effective $x$ and $Q^2$ values of the e--N
interaction, such that the virtual photon probes a modified quark
distribution compared to the stationary nucleons.  While the nucleus is built
from several constituents, it is common to consider only the contribution from
the nucleons as part of the 'conventional', and treat contributions from pions
and more exotic constituents such as virtual $\Delta$s, modified nucleons, or
hidden color configurations as additional contributions.

Akulinichev~\etal~\cite{Akulinichev:1985ij, Akulinichev:1985xq} included the
mean nucleon removal energy in their calculations in addition to the simple
binding and Fermi motion. They found that the experimental results can be
reproduced by inclusion of the average one-nucleon separation energy, but
their approach has been criticized for having incorrect normalization of
the spectral function~\cite{Frankfurt:1985ui}.

\textit{JRA: is "improper normalization of S(p)" the same as
failing to satisfy the momentum sum rule?}

\textit{JRA: Kulagin and Petti as more recent convolution approach
(presumably better than the 1985 versions, e.g. if they do convolution for
full 3D momentum distribution).  Includes removal energy based on single
particle separation energy, which accounts for roughly half of the observed
suppression of the high-$x$ nuclear structure function ratio.  Again, this
convolution yields a violation of the momentum sum rule.  In their analysis,
this is fixed by inclusion of an enhance pion field, although there are
several assumptions that go into this.}

...However, the contributions from nuclear pions are only significant at
low $x$, as they carry a momentum fraction of $m_{\pi}/m$ of the momentum of a
nucleon in the infinite momentum frame.  Thus, the structure functions
for $x>0.3$ should not be sensitive to the details of the nuclear pions (at
least not directly, pion model important in their extraction of off-shell
effects).

Another such calculation is by Benhar~\etal~\cite{benhar_emccalcnucmatt}.
Their binding-only contributions agrees with existing world data for the
medium heavy nuclei large $x$ values but undershoots the data at small $x$
values. The agreement is improved when they include the contributions of
``nuclear pions''. Another calculation by Marco~\etal~\cite{Marco_dis:1995}
uses realistic nucleon spectral functions and meson nucleus self energies to
get the size of the EMC effect outside of shadowing region. Their
calculations incorporate Fermi motion, binding, and pionic effects and uses an
interacting Fermi sea and the local density approximation to scale the results
from nuclear matter to finite nuclei. Here, also a better agreement with data
is reached when they use both nucleonic and mesonic (pions and rhos)
contributions as opposed to the nucleonic contribution alone.

%improved version of Marco's calculation~\cite{SajjadAthar:2009cr})}

\textit{JRA: The paragraph above was moved here from following section; may
need to be worked in better and have more (or less?) detail.}

\textit{JRA: need at least brief comment about Miller/Smith work which
claims that binding can't explain the EMC effect, but in a LC convolution
calculation with no binding (IIRC).}


Some calculations include more than just pions and nucleons.  A recent work
accounts for some of the deficit in the momentum-sum rule for the nucleons by
a modification to the Coulomb field of the
nucleus~\cite{Frankfurt_photon_pdf2010, Frankfurt:2012qs}.  If one is starting
from a convolution model which uses the separation energy, rather than the
average binding energy per nucleon, accounting for the momentum in the Coulomb
field simply accounts for loss of momentum from the nucleons; it does not
yield an additional suppression of the structure function at large $x$.
However, it would suggest that the modification to the nuclear pion field,
used to explain the deficit of the momentum sum rule, is overestimated in
heavy nuclei where the modification of the Coulomb field is more significant.
These and other works, e.g.~\cite{hen13}, also argue for the importance of
accounting for the difference between $x$, calculated with the proton mass,
and $x_A$ calculated with the mass of the nucleus.  This small shift in $x$
will yield a suppression of the nuclear structure function at large $x$, and
so this could be removed from the measured EMC effect. However, this effect is
also accounted for in calculations of the EMC ratios which include the binding
energy of the nucleus, so there is no need to apply a correction to the data
for this effect when comparing to such calculations.


%Because the pion is composed of a valance quark anti-quark pair, any
%model with such pion enhancement will naturally lead to an enhancement of
%anti-quarks at low $x$. In general, pion models describe the experimental data
%fairly well from $0.2<x<0.8$ but are less successful in explaining Drell-Yan
%data~\cite{Alde_DY_E772:1990}. These models~\cite{berger} predict a strong
%enhancement of the anti-quark distribution, which is not seen in the data.
%
%%JRA \textit{JRA: Excluded Drell-Yan discussion - more detail than we need, not
%%JRA sure what it adds to this paper}
%
% ...and I think that the berger calculation doesn't account for the pion
%mass properly (x_pi vs. x_p), so excluding it doesn't say much}.
%
% JRA: not sure that mathematical details of the generalized convolution
% (with pions) is required, and it begs the question about details for 
% other contributions, modified nucleons, etc....
%
%As mentioned previously, since there is nonzero probability of finding other
%hadrons in the nucleus (pions, deltas, multi-quark clusters \ldots)
%contribution from these hadron should also be included in the nuclear
%structure function calculation. Thus, the most general form of the convolution
%form can be written as:
%%
%\begin{equation}\label{conv3_eqn}
%F_2^A(x)=\sum_{i}{ \int_{x}^A dy~f_{i}^A(y) F_2^{i}\left(\frac{x}{y}\right)} ~.
%\end{equation}
%%
%For example, if we include the contribution from pions also with nucleons,
%then Eqn.~\ref{conv_eqn} becomes:
%%
%\begin{equation}\label{conv4_eqn}
%F_2^A(x)=\int_{x}^A dz~ f_{N}^A(z) F_2^{N}\left(\frac{x}{z}\right) +
%\int_{x}^A dy~f_{\pi}^A(y) F_2^{\pi}\left(\frac{x}{y}\right) ~,
%\end{equation}
%%
%where $F_2^{\pi}$ is the pion structure function and $f_{\pi}^A(y)$ is
%the momentum distribution of pions in nucleus. Pion models often assume that
%the nucleon structure function is unchanged by the nuclear medium, and the
%observed modifications in the nuclear structure functions are due to extra
%pions. Pions contribute at small $x$ values because in QPM they carry a
%fraction of $m_{\pi}/m$ of the momentum of a nucleon in the infinite momentum
%frame. This implies that if there are more pions in the bound nucleon compared
%to the free nucleon, their contribution is more significant in the low $x$
%region. Because the pion is composed of a valance quark anti-quark pair, any
%model with such pion enhancement will naturally lead to an enhancement of
%anti-quarks at low $x$. In general, pion models describe the experimental data
%fairly well from $0.2<x<0.8$ but are less successful in explaining Drell-Yan
%data~\cite{Alde_DY_E772:1990}. These models~\cite{berger} predict a strong
%enhancement of the anti-quark distribution, which is not seen in the data.


%Another approach recognizes that, since the nucleus is a dense system, there
%is a possibility that the valence quarks in the nucleus can form clusters of a
%color singlet state containing $6,9,12 \ldots$ quarks~\cite{jaffe83}. As in
%the case of pions, the momentum carried by the six quark cluster changes at
%the hadronic level. The success of the cluster model relies on the fact that a
%quark in a 6 or 9 quark bag has the possibility of carrying the momentum of
%two or three nucleons. Neglecting Fermi motion, the structure function of the
%nucleon (the conventional picture based on 3 quark state) vanishes for
%$x\ge1$. On the other hand, for a multi-quark configuration the structure
%function extends beyond $x=1$~\cite{geesaman95}.

Additional contributions that have been examined are virtual constituents of
the nucleus which are not present in a nucleon.  In the dense environment of
a nucleus, one may have color-singlet clusters of 6, 9,... valence
quarks~\cite{jaffe83, pirner11}.  The pdfs within such clusters can differ
significantly from the sum of a set of 2 or 3 individual nucleons, just 
yielding a modification to the nuclear pdf.  \textit{disfavored because
large contributions required to explain EMC region (?); can be examined
at larger $x$, as  multi-quark configuration the structure
function extends beyond $x=1$~\cite{geesaman95, sargsian03, arrington06,
fomin10}}.  It has also been suggested that hidden-color configurations may
contribute to the short-distance structure of the nucleus, which again could
potentially yield a small contribution with a significant modification to the 
nucleon pdfs.  

%rescaling models

Finally, some calculations invoke a modification to the internal structure
of individual nucleons within the dense medium of the nucleus.
Different rescaling models~\cite{close83,Nach_pirner, chanfray:1984} have been
proposed to explain the EMC effect, based on a change in the nucleon radius
due to partial deconfinement in the nuclear medium. In terms of QCD, a change
in confinement means a change in $Q^2$. Thus, QCD evolution starts at lower
$Q^2$ for a free nucleon, and, hence, the QCD radiative processes per nucleon
are larger in a bound nucleon than in a free nucleon.
%
% JRA: Eqn. was very confusing, and isn't really necessary given the level
% of discussion
%%
%\begin{equation}\label{rescale_eqn}
%F_2^A\left(x,Q^2\right) \sim F_2^{D}\left(x,\xi_A\left(Q^2\right),Q^2 \right) ~.
%\end{equation}
%%
In this case, scaling is referred to as ``dynamic'' because of the
evolution of the quark, anti-quark and gluon distributions.
Close~\etal~\cite{close85} shows that and increase in confinement size could
explain the data on a medium nucleus such as iron but fail to explain the data
for $x \gtorder 0.65$, since there is no inclusion of Fermi motion effects.

There are other models involving medium-modified nucleons that do not use
a rescaling of $Q^2$.  In such models the quark wave function of a nucleon is
modified by external fields provided by the surrounding nucleons. Quark-meson
coupling models~\cite{Saito:1994ki} include the effect of the nuclear medium
by allowing quarks in nucleons to interact via meson exchange and additional
vector and scalar fields. These models have been applied to the study the EMC
effect in unpolarized and polarized \cite{Cloet:2005rt, Cloet:2006bq}
structure functions, as well as other observables for nuclei and nuclear
matter~\cite{Saito:2005rv}. In addition, calculations for finite
nuclei~\cite{Cloet:2006bq} show a significant difference between the polarized
and unpolarized EMC effect. Recent work by Miller and Smith use a Chiral
soliton model to relate nucleon form factor modification~\cite{Smith:2004dn},
the EMC effect in polarized~\cite{Smith:2005ra} and
unpolarized~\cite{Smith:2002ci} structure functions.

%%JRA: \textit{JRA: Excluded isospin dependence; is it relevant enough?
%%JRA: if so, needs more detail (calcs. vs data), etc...}.

%There have also been suggestions a significant isospin-dependence for the EMC
%effect~\cite{Cloet:2006bq, hirai07, Dutta:2010pg}. In the model by
%Cloet~\etal, a neutron or proton excess in nuclei generates an isovector mean
%field which, through its coupling to the quarks in a bound nucleon, creates a
%shift in the quark distributions. The isospin dependence of the interaction
%leads to different degree of modification for the up and down quark
%distributions, yielding a difference in the EMC effect for these nuclei.

The initial results from our EMC ratios on light nuclei~\cite{seely09}
suggested that clustering effects may be important in generating the observed
structure function modification~\cite{Hirai_clusteringSF2010}.  One possibility
suggested by this result is the idea that the local environment of the struck
nucleon, rather than average nuclear density, is a driving force for much of
the modification to the nuclear structure function.  This does not directly
explain which of the models above may yield the best microscopic explanation
for the effect, but it suggests that the details of the nuclear structure
may need to be incorporated into quantitative evaluations of such effect.
Currently, most such calculations begin with a simple mean-field model or
infinite nuclear matter, and assume a simple scaling with density or some
other parameter rather than predicting the A dependence directly.  As such,
it has been difficult to determine which underlying physics best describes
the data, as the universal $x$-dependence and weak A dependence of the
earlier data from SLAC provided limited ability to constrain the models.
With the JLab data on light nuclei, more detailed calculations need to be
performed to test the A dependence associated with these models.



\subsection{Physics motivation behind E03-103}\label{motivation.ssec}

%%importance of light nuclei

The experiment reported here, JLab E03-103, was designed to precisely map out
the $x$, $Q^2$ and A-dependence of the inclusive electron scattering from
light to medium heavy nuclei, with emphasis on light nuclei and large $x$
region~\cite{E03103proposal}. Results for the EMC ratios for the light nuclei
have been reported in reference~\cite{seely09}. 

While the EMC effect has been well measured in heavy nuclei, the SLAC E139
ratios for $^4$He have large uncertainties and there are no measurements on
$^3$He in the valence region. Data on light nuclei are important in
understanding the microscopic origin of the EMC effect as they allow direct
comparison to detailed few-body calculations with minimal nuclear structure
uncertainties. Data on light nuclei can also help constrain nuclear effects in
the deuteron which are critical to the extraction of the neutron structure
function from measurements on the deuteron~\cite{whitlow92, Arrington:2008zh,
accardi10, accardi11, arrington12b}.  The light nuclei allow for better tests
of the $A$ dependence of the EMC effect, while also providing measurements of
nuclei more similar to the deuteron in mass and density.

In addition, studies of short-range correlations~\cite{frankfurt93,
arrington99, egiyan03, egiyan06, shneor07, subedi08, fomin12, arrington12a}
suggest that high-density configurations play an important role in nuclei,
which could potentially yield a modification of the nucleon structure function
in overlapping nucleons~\cite{geesaman95, arrington01, sargsian03,
arrington04, arrington06, fomin10}.  If two-body effects have a significant
contribution to the EMC effect, then the EMC effect could look different in
few-body nuclei than it does in heavy nuclei, where the effects may be
saturated. There were also models which predicted a very different $x$
dependence for the EMC effect for $A$=3,4~\cite{smirnov99, burov99,
benhar_priv, afnan03}, so the inclusion of light nuclei was considered
important as a way to look for two-body effects as a possible source of medium
modification in nucleon structure.


%importance of high x

Beyond the focus on light nuclei, E03-103 emphasized taking data at large $x$,
where Fermi motion and binding effect dominate. Because of the lack of data
in this region and the limited data for few-body nuclei, many calculations of
the EMC effect are performed for nuclear matter and extrapolated to lower
density when comparing to the nuclear parton distributions.  In such cases,
the important contributions of binding and Fermi motion are not modeled
in detail, making it difficult to test models of effects beyond the convolution
model.  

While many models mentioned in the previous section have had some
success, most are incomplete.  They may work only in a limited $x$ range,
conflict with limitations set by other measurements, or explain the data while
neglecting Fermi motion and binding. However, it is clear that the effects of
binding and Fermi motion are important and contribute over the entire $x$
region, not just at the largest $x$ values.  The large $x$ data are
particularly sensitive to these effects and to the details of nuclear
structure.  As such, precise high-$x$ data for both light and heavy nuclei can
help to constrain these effects.

