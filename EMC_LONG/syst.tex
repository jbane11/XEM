\section{Systematic uncertainties}\label{syst.sec}
%{\color{blue}{discussion, not sure whether we want to include big table with details or a smaller version of the table which summarizes the errors, CHECK ALL NUMBERS CAREFULLY}}
% see syst.041311.tex in this dir for the big table
%{\color{blue}{ CHECK ALL NUMBERS CAREFULLY}}

Statistical uncertainties arise from the random variation in the various
yields used to compute the cross sections. The total systematic uncertainty in
the cross section extraction is taken as the sum in quadrature of all
systematic uncertainties of the quantities that contribute to the cross
section. The components of the systematic uncertainty can be broadly divided
into two groups: point-to-point uncertainties and normalization uncertainties.
Point-to-point uncertainties are due to effects which may vary with time,
kinematic conditions, or detector location, and so their effect is 
(or at least can be) uncorrelated between different data points. Normalization
(scale) uncertainties affect the measurement globally (e.g., target
thickness).  Most corrections involve a mixture of point-to-point and
normalization uncertainties. The resulting overall uncertainty in the cross
section ratios is less than the total uncertainty in the cross section itself
because the many of the scale uncertainties and some point-to-point type
errors cancel in the ratios. Table~{\ref{syst_short_table} -- summarizes the
systematic uncertainties in extracting cross section ratios. The dominant
remaining contributions from the scale uncertainties are those associated with
the absolute target thicknesses, radiative and background corrections.  These
range from 1.5--2.0\% on the EMC ratios, and are provided for each target
ratio in the supplementary data tables~\cite{online_material}. Individual
contributions are discussed below.

Possible offsets in the beam energy, spectrometer momentum, and spectrometer
angle can yield errors in our extracted cross sections. The uncertainties
associated with these quantities are determined by calculating the cross
section at nominal kinematics, and comparing this to the cross section when
each of the kinematic variables are shifted by the uncertainty of that
variable. We use our model cross section for these studies. Note that the
kinematic uncertainties almost completely cancel in the cross section ratios.

The point-to-point uncertainty in beam charge measurement was estimated to
be $0.5\%$. This was obtained by studying the residuals of the measured
currents during the calibration procedure, mainly due to small drifts of the
BCM gain.  A scale uncertainty of $0.2\%$ was assumed for the charge measured,
due to the UNSER calibration.

%(Aji: looked at the impact of target purity and found that these are negligible for the EMC ratios ).

As mentioned in section~\ref{target.ssec}, thicknesses of the solid targets
were calculated using measurements of the mass and area of the targets.
Thicknesses of the cryotargets were computed from the target density and the
length of the cryogen in the path of the beam. The absolute uncertainty in the
$^2$H thickness is estimated to be 1.29\%. The normalization uncertainty due
to target thickness for the A/D cross section ratios are found to be $1.59\%$
and $1.23\%$ for the $^3$He and $^4$He targets respectively. For heavy nuclei,
this scale uncertainty is found to be between 0.5\% to 1\% for the target
ratios.  In addition to the nominal target densities, there are corrections 
associated with beam heating effects and fluctuations in the pressure and
temperature. The uncertainty associated with this correction comes from the
uncertainties in the fits to target luminosity scans. Though no boiling
correction is made in the case of the deuterium target, the uncertainty
arising due to boiling of this correction are still included in A/D ratios
for solid targets.  We assign a scale uncertainty of $0.24\% - 0.38\%$ for
the target ratios.

The scale uncertainty of the acceptance correction in the HMS was estimated to
be $1\%$ from the elastic cross section studies. This is a combination of the
uncertainties from the effect of position uncertainties in the target,
collimator, magnets, and detector package on the acceptance correction. The 
point-to-point uncertainty comes from comparison of model in the inelastic
region (where the cross sections does not vary significantly) to data, and is
estimated to be $1\%$. For the target ratios, the point-to-point uncertainty
was estimated to be $0.3\%$. For the cryotarget ratios, the scale uncertainty
was estimated to be $0.2\%$ and for solid target ratios this is estimated to
be $0.5\%$. This difference occurs because part of the effect cancels in
the cryotarget ratios.

The normalization uncertainty of the tracking efficiency is determined to be
$0.7\%$, mainly due to the limitations of the algorithm used for tracking.
Also a point-to-point uncertainty of $0.3\%$ is assigned to the tracking
efficiency in the target ratios.

%  [JRA: if it's really a 0.05%, then it's negligible in the scale.  Skip
%   text, since as long as table is OK, this level of detail doesn't matter
%
%Though the trigger efficiency was better than 0.992, a scale uncertainty of
%$0.05\%$ is assigned to account for the possible limitation of the trigger;
%Aji: Dave's table had 0 for A/D.
%
%(We assign a scale uncertainty of $0.06\%$ to the electronic dead time, mainly
%from the deviation of the measured value of $\tau$ from the expected value of
%60 ns; Aji: Dave's table had 0 for A/D). The point-to-point uncertainty in
%computer dead time depends on the trigger set up, $\pi/e$ ratio, etc. and is
%estimated to be $0.3\%$.

At very low $x$ values, the structure functions are expected to scale, and any
deviation is possibly due to the charge symmetric background (since this is
the dominant uncertainty for heavy nuclei at small $x$ and large scattering
angles). A comparison of 40 and 50 degree data suggests that scaling is
satisfied if the CSB varies by no more than $5\%$. A polynomial fit was made
to the charge symmetric background as a function of $x$, and $5\%$ of the
magnitude of the charge symmetric background is applied as the point-to-point
uncertainty in the charge symmetric background subtraction.

The uncertainty due to the model dependence in the radiative correction was
studied by varying the DIS and QE models independently. The change in cross
section was most pronounced in the low $x$ region ($x<0.4$). The relative
uncertainty in the cross section from the model dependence is estimated to be
$1\%$. For the helium target ratios, the point-to-point uncertainty is
estimated to be $0.5\%$ and we assign a scale uncertainty of $0.1\%$ due to
the difference in radiation length of the helium and deuterium targets. For
nuclei with $A>5$, the point-to-point uncertainty in the cross section ratios
is estimated to be $0.5\%$. Also a scale uncertainty of $1\%$ is assigned to
the cross section ratios of nuclei with large radiation length.

The effect of the model on the bin centering corrections was studied by
varying the shape of the model. This is done by supplying artificial $x$ and
$Q^2$ dependence as input to the individual DIS and QE pieces in the model
cross section. The variation was found to be most pronounced for the $x>$0.8
region, and we estimate a point-to-point uncertainty of $0.2\%$ for the
cross sections, and $0.1\%$ for the cross section ratios.

Uncertainties in the Coulomb corrections is mainly due to the knowledge of
the energy shift, $\Delta E$, used in the EMA calculation. We estimate this to
be known at the $10\%$ level. For the Au target at 40 degrees, this
uncertainty ranged from $0.5\%$ at low $x$ to $1.5\%$ at high $x$.

\begin{table}[htb]
% \begin{center}
\caption[]{Table shows the typical sources and magnitude of the systematic
uncertainties in extracting cross section ratios. These are added in
quadrature with the statistical uncertainties to get the total random
uncertainties.}
    \begin{tabular}{lcc}
	 \hline
	 \hline
	 Item & Absolute & $\delta R/R$ ($\pm$\%) \\
       & uncertainty($\pm$) &        \\
	 \hline
Beam Energy (offset)    & 5$\times$10$^{-4}$ 	& --   \\
Beam Energy (tgt-dep)	& 2$\times$10$^{-4}$ 	& 0.08    \\
HMS Momentum (offset)  	& 5$\times$10$^{-4}$ 	& --  \\
HMS Momentum (tgt-dep) 	& 2$\times$10$^{-4}$ 	& 0.0-0.12  \\
HMS angle (offset)   	& 0.5 mrad      	& -- \\
HMS angle (tgt-dep)	& 0.2 mrad		& 0.29-0.60 \\
Beam Charge       	& 0.5\%       	& 0.31 \\
Target Boiling     	& 0.45\%       	& 0.0-0.1 \\
End-cap Subtraction   	& 2--3\%       	& 0.28-0.45 \\
Acceptance       	& 1\%        	& 0.3 \\
Tracking Efficiency   	& 0.7\%       	& 0.3 \\
Trigger Efficiency   	& 0.3\%       	& 0.0 \\
Electronic Dead Time  	& 0.06\%       	& 0.0 \\
Computer Dead Time   	& 0.3\%       	& 0.3 \\
Charge Symmetric BG   	&          	& 0.0-1.0 \\ 
Coulomb corrections   	& 0.2\%       	& 0.1 \\
Pion Contamination   	& 0.2\%       	& 0.1 \\
Detector Efficiency   	& 0.2\%       	& 0.0 \\
Radiative Corrections  	& 1\%        	& 0.5 \\
Bin-centering      	& 0.2\%       	& 0.1 \\
    
\hline
\hline
    \end{tabular}
% \end{center}
 \label{syst_short_table}
\end{table}


